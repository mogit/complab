% small.tex
\documentclass{beamer}
\usetheme{Berkeley}
\begin{document}

\begin{frame}{AND GATE}
\begin{block}{}
    \begin{itemize}
	\item A logic gate is an idealized or physical device implementing a Boolean function, that is, it performs a logical operation on one or more logic inputs and produces a single logic output.
	\item AND gate takes two inputs and gives output as low(0) whenever any of it's input is low(0).   
	\item It is represented as \alert{C=A.B} 
    \end{itemize}
  \end{block}
\pause
\begin{block}{Truth Table}
\begin{tabular}{|c|c||c|}
\hline
 \textbf{A} &
\textbf{B} & \textbf{C} \\
\hline
\hline
 0 & 0 & 0 \\
\hline
 0 & 1 & 0 \\
\hline
 1 & 0 & 0 \\
\hline
 1 & 1 & 1 \\
\hline
\end{tabular}
\end{block}
\end{frame}
\begin{frame}{OR GATE}
\begin{itemize}
  \item OR gate takes two inputs and gives output as high(1) whenever any of it's input is high(1).
  \item It is represented as C=A+B
\end{itemize}

\begin{tabular}{|c|c||c|}
\hline
 \textbf{A} &
\textbf{B} & \textbf{C} \\
\hline
\hline
 0 & 0 & 0 \\
\hline
 0 & 1 & 1 \\
\hline
 1 & 0 & 1 \\
\hline
 1 & 1 & 1 \\
\hline
\end{tabular}
\end{frame}
\begin{frame}{NOT GATE}
\begin{itemize}
  \item NOT gate takes one inputs and gives complementary output.
  \item It is represented as C=A
\end{itemize}

\begin{tabular}{|c|c|}
\hline
 \textbf{A} &
\textbf{C} \\
\hline
\hline
 1 & 0 \\
\hline
 0 & 1 \\
\hline
\end{tabular}
\end{frame}
\begin{frame}{De Morgan's laws}
\end{frame}

\end{document}
